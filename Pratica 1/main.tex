\documentclass{beamer}
\usepackage{amsmath}
\usepackage{amsfonts}
%\usepackage[utf-8]{inputenc}
\usepackage{tikz}
\usepackage[portuguese]{babel}

\author[P. Fagandini]{Paulo Fagandini}
\title[Parte I]{Parte I: Aula Pr\'atica}
\institute[ISCAL - IPL]{Lisbon Accounting and Business School}
\date{}

\begin{document}

\maketitle

\begin{frame}{Ex 16, Caderno 1}
    A Amélia é a namorada do Nicolau. Ambos adoram fazer um demorado Brunch ao
domingo. Gostam de incluir tostas $(x)$ e ovos cozinhados $(y)$ de várias maneiras, mas não
querem demorar mais de 30min na sua preparação, preferindo degustar
preguiçosamente… Naquele período de tempo, se a Amélia fizer apenas tostas,
consegue fazer 3, se só fizer ovos, consegue executar 6 receitas; por seu lado, se o
Nicolau apenas fizer tostas, consegue fazer 2, mas se só fizer ovos, consegue elaborar 8
receitas. Admita que as capacidades culinárias destes agentes económicos são bem
representadas por modelos lineares.

\end{frame}

\begin{frame}{Ex 16, Caderno 1}
    a) Apresente as Fronteiras de Possibilidades de Produção da Amélia e do Nicolau
e interprete os respectivos declives.

    \vspace{0.25in}

    \begin{columns}
        \begin{column}{0.5\textwidth}
            Am\'elia

            \begin{tikzpicture}[scale = 0.5]%, every node/.style={scale=1}]
                
                \draw[->] (0,0) -- (4.5, 0)node[below]{$x$};
                \draw[->] (0,0) -- (0, 8.5)node[left]{$y$};

                \onslide<2-> {\filldraw[red] (3, 0) circle (5pt) node[below] {3};}
                \onslide<3-> {\filldraw[red] (0, 6) circle (5pt) node[left] {6};}

                \onslide<4-> {\draw[red] (3,0) -- (0,6);}
            \end{tikzpicture}
            \onslide<4->{FPP\textsubscript{a}:} \onslide<5->{\(y=6 - \frac{6}{3}x = 6-2x\)}
        \end{column}

        \begin{column}{0.5\textwidth}
            Nicolau

            \begin{tikzpicture}[scale = 0.5]%, every node/.style={scale=1}]
                \draw[->] (0,0) -- (4.5, 0)node[below]{$x$};
                \draw[->] (0,0) -- (0, 8.5)node[left]{$y$};

                \onslide<6->{\filldraw[red] (2, 0) circle (5pt) node[below] {2};}
                \onslide<7->{\filldraw[red] (0, 8) circle (5pt) node[left] {8};}

                \onslide<8->{\draw[red] (2,0) -- (0,8);}
            \end{tikzpicture}
            \onslide<9->{FPP\textsubscript{n}:} \onslide<10->{\(y=8 - \frac{8}{2}x = 8-4x\)}
        \end{column}
    \end{columns}

\end{frame}

\begin{frame}

    b) Qual deles tem vantagem comparativa na preparação de receitas de ovos? E na
preparação de tostas? Devem ambos preparar tostas e ovos, ou deve cada um
especializar-se numa das actividades? Explique, referindo como se pode
interpretar neste caso a possibilidade teórica de ``comércio'' entre eles.

\vspace{0.25in}

\begin{columns}
    \centering
    \begin{column}{0.5\textwidth}
        \centering
        \onslide<2->{\(FPP_a: y = 6 - 2x\)}

        \onslide<3->{\(2\)}
    \end{column}

    \begin{column}{0.5\textwidth}
        \centering
        \onslide<2->{\(FPP_n: y = 8 - 4x\)}

        \onslide<3->{\(4\)}
    \end{column}
\end{columns}

\vspace{0.25in}

\onslide<4->{Para quem \'e mais barato produzir tostas $(x)$?} \onslide<5->{para Am\'elia! Pelo que o Nicolau deve
se especializar na preparação de ovos $(y)$.}
    
\vspace{0.15in}

\onslide<6->{Os termos de troca para haver comércio devem encontrarse em \([2, 4]\)}

\end{frame}

\begin{frame}
    c) Admita que os termos de troca se fixaram no centro do intervalo de valores
possíveis e que terá referido na alínea anterior. Apresente a fronteira das
possibilidades de consumo e explique como se pode definir um plano de
produção e trocas para que se atinjauma situação em que ambos consomem no
Brunch 1.5 tostas cada um. Nesse caso, quantos ovos consumiriam?

\vspace{0.25in}
\begin{itemize}
    \item<2->{O ponto m\'edio de \([2, 4]\) \'e 3, ou seja 3 ovos por tosta.}
    \item<3->{Nicolau, como se especializou no $y$, tem a mesma ordenada na origem,
    Assim, a FPC\textsubscript{n}: $y = 8 - 3x$}
    \item<4->{Am\'elia, tem 3 tostas, pelo que poderia troca-las por at\'e 9 ovos ($3\times 3$).
    Assim, a sua FPC\textsubscript{a}: $y = 9 - 3x$}
    \item<5-> Se trocam 3 tostas por ovo, Am\'elia podia dar 1.5 tosta ao Nicolau E
    em troca iria receber $1.5 \times 3 = 4.5$ ovos.
    \item<6-> Finalmente Am\'elia consume $(1.5, 4.5)$, e Nicolau $(1.5, 3.5)$.
\end{itemize}

\end{frame}

\begin{frame}

    \begin{columns}
        \begin{column}{0.5\textwidth}
            Am\'elia

            \begin{tikzpicture}[scale = 0.5]
                
                \draw[->] (0,0) -- (5, 0)node[below]{$x$};
                \draw[->] (0,0) -- (0, 9.5)node[left]{$y$};

                \filldraw[red] (3, 0) circle (5pt) node[below] {3};
                \filldraw[red] (0, 6) circle (5pt) node[left] {6};

                \draw[red] (3,0) -- (0,6);
                \draw[blue] (3, 0) -- (0, 9)node[left]{$9$};

                \filldraw[blue] (1.5, 4.5) circle (5pt) node[above right]{$(1.5, 4.5)$};

            \end{tikzpicture}
        \end{column}

        \begin{column}{0.5\textwidth}
            Nicolau

            \begin{tikzpicture}[scale = 0.5]
                \draw[->] (0,0) -- (5, 0)node[below]{$x$};
                \draw[->] (0,0) -- (0, 9.5)node[left]{$y$};

                \filldraw[red] (2, 0) circle (5pt) node[below] {2};
                \filldraw[red] (0, 8) circle (5pt) node[left] {8};

                \draw[blue] (0, 8) -- ({8/3}, 0) node[below right] {$2.(6)$};
                \draw[red] (2,0) -- (0,8);

                \filldraw[blue] (1.5, 3.5) circle (5pt) node[above right]{$(1.5, 3.5)$};
            \end{tikzpicture}
        \end{column}
    \end{columns}

\end{frame}

\begin{frame}
    d) O Baltazar é muito amigo do Nicolau e da Amélia. Se ele também quiser
    frequentar o Brunch de Domingo, não será eficiente levar também tostas e ovos
    para juntar aos dos seus amigos e todos comerem juntos, será melhor levar taças
    de fruta, que ele prepara bastante bem. Explique como funcionaria este modelo,
    à luz do conceito de vantagem comparativa e vantagens de comércio, caso
    tenhamos 3 produtos para troca.

    \vspace{0.25in}

    \onslide<2->{
        Certo, porque se tem ventagem comparativa na produ\c c\~ao de um tereceiro bem
        assim todos podem estar melhor. Suponha que tem ventagem comparativa na
        produ\c c\~ao de $x$ ou $y$, assim sempre algum dos 3 n\~ao iria ganhar
        nada com com\'ercio com algum dos outros. Assim com um terceiro bem,
        todos podem trocar com todos e estar melhor.
    }
\end{frame}


\begin{frame}
    e) Os Brunchs de domingo são um sucesso! A Amélia pensa em abrir um
restaurante. Determine o custo de oportunidade dessa decisão, caso seja
necessário a Amélia abdicar do seu emprego onde ganha 20 mil euros líquidos
por ano, sabendo que terá de investir 10 mil euros em remodelação do espaço,
maquinaria e licenças, para poder abrir o seu restaurante de Brunchs. Explique a
diferença entre custo de oportunidade e despesa (no sentido contabilístico).
Qual o volume de facturação que fará do negócio uma escolha racional?

\vspace{0.25in}

\onslide<2->{
    \[CO=20 + 10 = 30\]
}

Ter\'a de faturar mais de 30 mil para que seja uma escolha racional.

\end{frame}
\end{document}