\documentclass[table,handout]{beamer}
\usepackage[utf8]{inputenc}
\usepackage{xcolor}
\usepackage{tcolorbox}
\usepackage{adjustbox}
\usepackage{multicol}
\usepackage{multirow}
\usepackage{eurosym}
\usepackage{amsmath}
\usepackage{ragged2e}
\usepackage{tikz}

%\usecolortheme{owl}

% \usepackage[scaled]{helvet}
% \renewcommand\familydefault{\sfdefault} 
% \usepackage[T1]{fontenc}

\usepackage{tgadventor}
\renewcommand*\familydefault{\sfdefault} %% Only if the base font of the document is to be sans serif
\usepackage[T1]{fontenc}

\usetikzlibrary{spy,shapes,arrows,positioning,decorations.pathreplacing}

\tikzset{hide on/.code={\only<#1>{\color{white}}}}
\definecolor{dark_green}{rgb}{0.0, 0.5, 0.0}

\pgfdeclarelayer{bg}    % declare background layer
\pgfsetlayers{bg,main}  % set the order of the layers (main is the standard layer)

\apptocmd{\frame}{}{\justifying}{}

\definecolor{iscal_color}{HTML}{641242} % ISCAL

\setbeamertemplate{footline}{
	\hspace{0.05\textwidth}
	\raisebox{3ex}{\insertshortauthor{}}\hfill
	\raisebox{3ex}{\insertframenumber{}/\inserttotalframenumber} \hfill
	{\includegraphics[height=0.08\textheight]{../visual material/logo.png}}
	\hspace{0.05\textwidth}
}

\tikzset{
  invisible/.style={opacity=0},
  visible on/.style={alt={#1{}{invisible}}},
  alt/.code args={<#1>#2#3}{%
    \alt<#1>{\pgfkeysalso{#2}}{\pgfkeysalso{#3}} % \pgfkeysalso doesn't change the path
  },
}

\title{Microeconomia}
\subtitle{Capitulo 1 : Princ\'ipios fundamentais}
\author[]{}
\institute[ISCAL]{\includegraphics[height=0.10\textheight]{../visual material/logo_eng_full.png}}
\date{}

\setbeamertemplate{navigation symbols}{}

\setbeamercolor{title}{fg = iscal_color}
\setbeamercolor{subtitle}{fg = iscal_color}
\setbeamercolor{frametitle}{fg = white, bg = iscal_color}

\hypersetup{linkcolor=iscal_color, colorlinks=true}

\AtBeginSection{\frame{\sectionpage}}
\renewcommand{\sectionname}{Parte}

\begin{document}

{
\setbeamertemplate{footline}{}
\begin{frame}
	\maketitle
\end{frame}
}

\begin{frame}{Conte\'udos}
  \tableofcontents
\end{frame}

\section{Da cadeira}

\begin{frame}
  \begin{itemize}
    \item Professor: Paulo Fagandini
    \item e-mail: pfagandini@iscal.ipl.pt
  \end{itemize}

  \vspace{1cm}

  \begin{itemize}
    \item Professor Regente: Ant\'onio Morgado
    \item e-mail: ajmorgado@iscal.ipl.pt
  \end{itemize}  
\end{frame}

\begin{frame}
  Modalidade do curso:
  \begin{itemize}
    \item<1-> Aulas focadas na parte matem\'atica, frequentemente a mais complexa para os alunos.
    \item<2-> Conceitos mais b\'asicos e defini\c c\~oes nas refer\^encias bibliogr\'aficas obligat\'orias.
  \end{itemize}
  \onslide<3->{Avalia\c c\~ao:}
  \begin{itemize}
    \item<4-> 2 testes de escolha m\'ultipla nas salas de computadores num turno de dia Sabado. Cada teste vale 50\%.
    \begin{itemize}
      \item<4-> 1 teste, 26 Outubro.
      \item<4-> 2 teste, 14 Dezembro.
    \end{itemize}
    \item<5-> Nota m\'inima de 7 valores para se manter em avalia\c c\~ao cont\'inua.
    \item<6-> Avalia\c c\~ao cont\'inua n\~ao \'e obrigat\'oria, pelo que n\~ao h\'a outras datas para as avalia\c c\~oes.
  \end{itemize}
\end{frame}

\section{Conceitos Fundamentais}
\begin{frame}
	\frametitle{A Economia como ci\^encia}

	\begin{center}
		\textbf{Problema Econ\'omico}
	\end{center}

	\vspace{0.2cm}

	\pause

	decidir \textbf{o que} produzir, \textbf{como} e \textbf{para quem}, utilizando recursos escassos, pass\'iveis de utiliza\c c\~oes alternativas, num contexto de n\~ao saciedade (necessidades ilimitadas).

\end{frame}

\begin{frame}
	\frametitle{A Economia como ci\^encia}
	Segundo Lionel Robbins (1935):

	\begin{tcolorbox}[colback=red!5,colframe=red!40!black,title=Economia]
		Ci\^encia que estuda o comportamento humano como uma rela\c c\~ao entre fins e meios escassos que t\^em usos alternativos
	\end{tcolorbox}

\end{frame}

\begin{frame}
	\frametitle{A Economia como ci\^encia}

	Podemos dividir a Economia em duas grandes \'areas:\pause

	\begin{itemize}
		\item \textbf{Microeconomia} estuda o comportamento e interac\c c\~ao de consumidores e produtores, enquanto indiv\'iduos isolados, que se encontram num mercado. \pause
		\item \textbf{Macroeconomia} estuda o desempenho da economia \`a escala nacional. Analisa vari\'aveis agregadas como o rendimento, o emprego e o investimento. Estuda fen\'omenos com a infla\c c\~ao e os ciclos econ\'omicos.
	\end{itemize}
\end{frame}

\begin{frame}
	\frametitle{An\'alise Econ\'omica}
	Dois importantes vertentes de an\'alise:\pause
	\begin{itemize}
		\item Economia Positiva: an\'alise cient\'ifica, objectiva, com conclus\~oes demonstr\'aveis e verific\'aveis.\pause
		\item Economia Normativa: an\'alise subjectiva, influenciada por ju\'izos de valor, em fun\c c\~ao de preceitos pol\'iticos, \'eticos ou morais.
	\end{itemize}
\end{frame}

\begin{frame}
	\frametitle{N\~ao confundir \textbf{E}conomia com \textbf{e}conomia}

	\begin{itemize}
		\item {\color{blue}\textbf{E}conomia} diz respeito \`a ci\^encia.
		\item {\color{blue}\textbf{e}conomia} \'e um agregado de ``agentes econ\'omicos'' (indiv\'iduos que tomam decis\~oes) que interagem em determinado espa\c co.
	\end{itemize}

\end{frame}

\begin{frame}
	\frametitle{A escassez}

	\begin{itemize}

		\item O que seria o mundo sem escassez?\pause
		
		\vspace{0.2cm}

		N\~ao haveria necessidade de escolher entre utiliza\c c\~oes alternativas para um recurso, porque ele existiria em quantidades ilimitadas... \pause

		\item A escassez obriga a que se fa\c cam \textbf{escolhas}, levando a um \textbf{trade-off}: para se ter uma utiliza\c c\~ao de um recurso, prescinde-se (total ou parcialmente) de outra utiliza\c c\~ao alternativa

	\end{itemize}

\end{frame}

\begin{frame}
	\frametitle{Custos Econ\'omicos}

	O custo econ\'omico de utiliza\c c\~ao de um recurso \'e o custo de oportunidade.\pause
	
	\begin{tcolorbox}[colback=red!5,colframe=red!40!black,title = Custo de Oportunidade]
		Valor gerado por um recurso na sua melhor utiliza\c c\~ao alternativa.
	\end{tcolorbox}\pause

	O \emph{Custo de Oportunidade} representa, portanto, o valor que os agentes econ\'omicos atribuem \`a melhor alternativa de que prescindem quando efetuam uma escolha.

\end{frame}

\begin{frame}
	\begin{center}	
	{\huge Qual \'e o custo de oportunidade da utiliza\c c\~ao de um recurso ilimitado?}
	\end{center}
\end{frame}

\begin{frame}
	\frametitle{Exemplo}
	\begin{itemize}
		\item O Jo\~ao tem um Pr\'edio Rural que pode vender por \euro1,000 no mercado, mas pagaria \euro100 de imposto sobre mais-valias de im\'oveis.\pause
		\item Se plantar eucaliptos, pode ter um rendimento de \euro1,800 por ano, mas ter\'a de investir \euro1,100 no cultivo e tratamento das \'arvores.
		\item Se optar por plantar eucaliptos, qual o custo de oportunidade da decis\~ao?
	\end{itemize}
\end{frame}

\begin{frame}
	\frametitle{Exemplo (cont.)}
	O \textbf{custo de oportunidade de plantar} eucaliptos, ser\'a o valor que o Jo\~ao conseguiria ter se optasse pela alternativa, vender:
	\begin{align}
		\textup{\euro} 1,000 - \textup{\euro} 100 + \textup{\euro} 1,100 = \textup{\euro} 2,000
	\end{align}
\end{frame}

\begin{frame}
	\frametitle{Exemplo (cont.)}
	O \textbf{custo de oportunidade de plantar} eucaliptos, ser\'a o valor que o Jo\~ao conseguiria ter se optasse pela alternativa, vender:
	\begin{align}
		\underbrace{\textup{\euro} 1,000 - \textup{\euro} 100}_{\text{Excedente na alternativa}} + \overbrace{\textup{\euro} 1,100}^{\text{Despesa que n\~ao teria na alternativa}} = \textup{\euro} 2,000
	\end{align}
\end{frame}

\begin{frame}
	\frametitle{Observa\c c\~ao}

	Qual a rela\c c\~ao entre custo de oportunidade de uma escolha e a despesa com a sua aquisi\c c\~ao? \pause

	\begin{itemize}
		\item A despesa com a aquisi\c c\~ao pode ser considerada um custo contabil\'istico... (no caso da planta\c c\~ao, \euro 1,100) \pause
		\item O custo de oportunidade \'e algo mais do que isso... (\euro 1,100 + excedente da melhor alternativa)
	\end{itemize}

\end{frame}

\begin{frame}
	\frametitle{Exemplo (cont.)}
	\begin{itemize}
		\item E se o Jo\~ao optar por vender o seu terreno?\pause
		\item O \textbf{custo de oportunidade da venda}, ser\'a o valor que o Jo\~ao conseguiria ter se optasse pela alternativa, plantar eucaliptos:\pause
		\begin{align}
			\underbrace{\textup{\euro}1,800 - \textup{\euro}1,100}_{\text{Excedente na alternativa}} + \overbrace{\textup{\euro}100}^{\text{Despesa que n\~ao teria na alternativa}} = \textup{\euro} 800
		\end{align}
	\end{itemize}
\end{frame}

\begin{frame}
	\begin{center}	
	{\huge Qual a decis\~ao \'otima? \pause \vspace{0.5cm} ... Racionalidade!}
	\end{center}
\end{frame}

\begin{frame}
	\frametitle{Racionalidade}

	A decis\~ao racional ser\'a aquela op\c c\~ao para a qual o custo de oportunidade \'e inferior ao benef\'icio bruto nessa op\c c\~ao. No exemplo: \pause

	\begin{itemize}
		\item Custo de oportunidade de vender = \pause \euro 800
		\item Benef\'icio bruto da venda = \pause \euro 1,000 \pause
	\end{itemize}

	\vspace{0.2cm}

	\begin{itemize}
		\item Custo de oportunidade de plantar =\pause \euro 2,000\pause
		\item Benef\'icio bruto da planta\c c\~ao =\pause \euro 1,800
	\end{itemize}
\end{frame}

\begin{frame}
	\frametitle{Racionalidade}
	A decis\~ao racional ser\'a, portanto, vender o terreno, pois \'e aquela para a qual o custo de oportunidade \'e inferior ao benef\'icio bruto:
	\begin{align}
		\underbrace{\textup{\euro} 1,800 - \textup{\euro} 1,100 + \textup{\euro} 100}_{\text{Custo de Oportunidade da venda}} < \overbrace{\textup{\euro} 1,000}^{\text{Benef\'icio bruto da venda}}
	\end{align}
\end{frame}

\begin{frame}
	\frametitle{Racionalidade}
	A desigualdade anterior pode ser escrita de outras duas formas, {\color{blue}equivalentes entre si}:
	\begin{align}
		\overbrace{\textup{\euro} 1,800 - \textup{\euro} 1,100 + \textup{\euro} 100}^{\text{Custo de Oportunidade da venda}} &< \overbrace{\textup{\euro} 1,000}^{\text{Benef\'icio bruto da venda}} \\ 
		\overbrace{\textup{\euro} 1,800 - \textup{\euro} 1,100}^{\text{Excedente, se plantar}} &< \overbrace{\textup{\euro} 1,000 - \textup{\euro} 100}^{\text{Excedente, se vender}} \\ 
		\overbrace{\textup{\euro} 1,800 - \textup{\euro} 1,000}^{\text{Benef\'icio marginal, se plantar}} &< \overbrace{\textup{\euro} 1,100 - \textup{\euro} 100}^{\text{Custo marginal, se plantar}} 
	\end{align}
\end{frame}

\begin{frame}
	\frametitle{Racionalidade}
	Temos, ent\~ao, tr\^es formas equivalentes de verificar racionalidade. Uma decis\~ao \'e racional se:
	\begin{itemize}
		\item O seu custo de oportunidade for inferior ao seu benef\'icio bruto;
		\item Se o seu excedente for o maior;
		\item Se o seu $Bmg$ for superior ao $Cmg$ (an\'alise custo-benef\'icio)
	\end{itemize}
\end{frame}
\section{Racionalidade e An\'alise Custo-Benef\'icio}
\input{lecture_2}
\section{Fronteira das Possibilidades da Produ\c c\~ao}
\input{lecture 4}
% \section{FPP - Modelos n\~ao lineares}
% \input{lecture 5}

\end{document}