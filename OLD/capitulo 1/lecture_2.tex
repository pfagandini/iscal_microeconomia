\begin{frame}
	\frametitle{An\'alise custo-benef\'icio}

	A compara\c c\~ao entre custos marginais e benef\'icios marginais \'e particularmente \'util quando \'e necess\'ario escolher a quantidade de um recurso que est\'a a ser utilizada ou a quantidade de um bem que se est\'a a produzir:

	\vspace{0.2cm}

	\pause

	Valer\'a a pena aumentar a quantidade enquanto o benef\'icio marginal (benef\'icio adicional por mais uma unidade) for superior ao custo marginal (custo adicional por essa unidade)

\end{frame}

\begin{frame}
	\frametitle{Exemplo}

	\begin{itemize}
		\item Um produtor de P\^era Rocha do Oeste precisa decidir que quantidade de p\^era deve colher nos seus pomares. Se colher mais, consegue vender mais, mas tamb\'em tem mais custos.
		\item As receitas e os custos s\~ao de acordo como quadro seguinte:
	\end{itemize}

\end{frame}

\begin{frame}
	\frametitle{Exemplo}

	\begin{table}
		\rowcolors{1}{red!5}{red!20}
		\begin{tabular}{p{1.8cm}p{1.8cm}p{1.8cm}p{1.8cm}p{1.8cm}}
			Quantidade (10s caixas) & Receitas (benef\'icio) & Benef\'icio marginal & Custos & Custo marginal \\
			\hline\hline
			10 & \euro 100 & & \euro 80 & \\
			11 & \euro 109 & & \euro 85 & \\
			12 & \euro 117 & & \euro 92 & \\
			13 & \euro 124 & & \euro 100 & \\
			14 & \euro 130 & & \euro 110 & 
		\end{tabular}
	\end{table}
\end{frame}

\begin{frame}
	\frametitle{Exemplo}

	\begin{table}
		\rowcolors{1}{red!5}{red!20}
		\begin{tabular}{p{1.8cm}p{1.8cm}p{1.8cm}p{1.8cm}p{1.8cm}}
			Quantidade (10s caixas) & Receitas (benef\'icio) & Benef\'icio marginal & Custos & Custo marginal \\
			\hline\hline
			10 & \euro 100 & - & \euro 80 & \\
			11 & \euro 109 & \euro 9 & \euro 85 & \\
			12 & \euro 117 & \euro 8 & \euro 92 & \\
			13 & \euro 124 & \euro 7 & \euro 100 & \\
			14 & \euro 130 & \euro 6 & \euro 110 & 
		\end{tabular}
	\end{table}
\end{frame}


\begin{frame}
	\frametitle{Exemplo}

	\begin{table}
		\rowcolors{1}{red!5}{red!20}
		\begin{tabular}{p{1.8cm}p{1.8cm}p{1.8cm}p{1.8cm}p{1.8cm}}
			Quantidade (10s caixas) & Receitas (benef\'icio) & Benef\'icio marginal & Custos & Custo marginal \\
			\hline\hline
			10 & \euro 100 & - & \euro 80 & - \\
			11 & \euro 109 & \euro 9 & \euro 85 & \euro 5 \\
			12 & \euro 117 & \euro 8 & \euro 92 & \euro 7 \\
			13 & \euro 124 & \euro 7 & \euro 100 & \euro 8 \\
			14 & \euro 130 & \euro 6 & \euro 110 & \euro 10
		\end{tabular}
	\end{table}

	$Bmg = \frac{\Delta B}{\Delta Q}$, $Cmg = \frac{\Delta C}{\Delta Q}$

\end{frame}

\begin{frame}
	\frametitle{Exemplo}

	\begin{table}
		\rowcolors{1}{red!5}{red!20}
		\begin{tabular}{p{1.8cm}p{1.8cm}p{1.8cm}p{1.8cm}p{1.8cm}}
			Quantidade (10s caixas) & Receitas (benef\'icio) & Benef\'icio marginal & Custos & Custo marginal \\
			\hline\hline
			10 & \euro 100 & - & \euro 80 & - \\
			11 & \euro 109 & \euro 9 & \euro 85 & \euro 5 \\
			12 & \euro 117 & \euro 8 & \euro 92 & \euro 7 \\
			13 & \euro 124 & \euro 7 & \euro 100 & \euro 8 \\
			14 & \euro 130 & \euro 6 & \euro 110 & \euro 10 \\
			16 & \euro 140 & \euro 5 & \euro 132 & \euro 11
		\end{tabular}
	\end{table}
\end{frame}


\begin{frame}
	\frametitle{Exemplo}

	\begin{table}
		\rowcolors{1}{red!5}{red!20}
		\begin{tabular}{p{1.6cm}p{1.6cm}p{1.6cm}p{1.6cm}p{1.6cm}p{1.6cm}}
			Quantidade (10s caixas) & Receitas (benef\'icio) & Benef\'icio marginal & Custos & Custo marginal & Lucro \\
			\hline\hline
			10 & \euro 100 & - & \euro 80 & - & \euro 20\\
			11 & \euro 109 & \euro 9 & \euro 85 & \euro 5  & \euro 24 \\
			\textbf{12*} & \textbf{\euro 117} & \textbf{\euro 8} & \textbf{\euro 92} & \textbf{\euro 7} & \euro 25 \\
			13 & \euro 124 & \euro 7 & \euro 100 & \euro 8 & \euro 24 \\
			14 & \euro 130 & \euro 6 & \euro 110 & \euro 10 & \euro 20 \\
			16 & \euro 140 & \euro 5 & \euro 132 & \euro 11 & \euro 12
		\end{tabular}
	\end{table}
\end{frame}

\begin{frame}
	\frametitle{Bmg e Cmg}

	\begin{figure}
		\centering
		\def\a{1.5}
		\def\b{2.5}
		\def\c{0.14}
		\begin{tikzpicture}[spy using outlines={red, circle, magnification=4, size=25 * 4,
                          connect spies}]
			\draw[->] (-0.1,0) -- (4.2,0) node[below right] {$Q$};
	 		\draw[->] (0,-0.1) -- (0,4.2) node[above left] {$Custo$};
	 		\draw[dashed] (0,{(\a-1)^2/2 + 0.5}) node[left] {92} -- (\a,{(\a-1)^2/2 + 0.5}) -- (\a,0) node [below] {12};
	 		\draw[dashed] (0,{(\b-1)^2/2 + 0.5}) node[left] {100} -- (\b,{(\b-1)^2/2 + 0.5}) -- (\b,0) node [below] {13};
	 		\onslide<2->{
	 			\draw[red,thick, ->] (\a,{(\a-1)^2/2 + 0.5}) -- ({(\a+\b)/2},{(\a-1)^2/2 + 0.5}) node [below] {$\Delta Q = 1$} -- (\b,{(\a-1)^2/2 + 0.5});
	 			\draw[red,thick,->] (\b,{(\a-1)^2/2 + 0.5}) -- (\b,{(((\a-1)^2/2 + 0.5)+((\b-1)^2/2 + 0.5))/2}) node [right] {$\Delta C = 8$} -- (\b,{(\b-1)^2/2 + 0.5});
	 		}
	 		\onslide<3->{
	 			\draw[blue,fill=blue!20] (\a,{(\a-1)^2/2 + 0.5}) node [above right] {$\alpha$}-- (\b,{(\a-1)^2/2 + 0.5}) -- (\b,{(\b-1)^2/2 + 0.5});
	 			\draw[smooth,variable=\x,domain=1:3.5] plot ({\x},{(\x-1)^2/2 + 0.5});
	 		}
	 		\onslide<4->{
	 			\draw[orange] ({\a+\c},{(\a-1)^2/2 + 0.5}) -- ({\b+\c},{(\b-1)^2/2 + 0.5});
	 		}
	 		\only<5->{\spy on (1.9,1) in node[overlay] at (6, 4);}
		\end{tikzpicture}
	\end{figure}
\end{frame}

\begin{frame}
	\frametitle{Bmg e Cmg}
	\begin{align*}
		\max\ &B-C\\
		B'-C' &= 0 \Rightarrow \\
		B'&= C' \Rightarrow \\
		 Bmg&=Cmg
	\end{align*}
\end{frame}

\begin{frame}
	\frametitle{Conclus\~oes}

	\begin{itemize}
		\item An\'alise marginal \'e o c\'alculo da varia\c c\~ao de uma vari\'avel por unidade adicional de outra \pause
		\item An\'alise marginal corresponde, portanto, a uma taxa de varia\c c\~ao m\'edia \pause
		\item Uma taxa de varia\c c\~ao m\'edia \'e uma derivada \pause
		\item O crit\'erio custo-benef\'icio corresponde \`a condi\c c\~ao de primeira ordem de m\'aximo
	\end{itemize}	

\end{frame}

\begin{frame}
	\frametitle{Racionalidade}

	\begin{itemize}
		\item Por vezes, os agentes econ\'omicos n\~ao tomam decis\~oes racionais, porque a racionalidade \'e limitada, j\'a que:\pause
			\begin{enumerate}
				\item A realidade \'e muito complexa \pause
				\item A capacidade cognitiva dos indiv\'iduos \'e limitada \pause
				\item A informa\c c\~ao \'e frequentemente incompleta \pause
			\end{enumerate}

		\item Mas admitiremos que os indiv\'iduos s\~ao racionais e que reagem a incentivos ... Escolhas racionais s\~ao escolhas eficientes.

	\end{itemize}


\end{frame}

\begin{frame}
	\frametitle{Efici\^encia}

	\pause
	\begin{itemize}
		\item Efici\^encia (no sentido de Pareto) significa n\~ao poder melhorar a situa\c c\~ao de um agente econ\'omico sem piorar a situa\c c\~ao de outro...\pause
		\item Em geral, todas as escolhas eficientes t\^em subjacente um \emph{trade-off}, ou seja uma situa\c c\~ao de escolha em que para ter mais de uma op\c c\~ao \'e preciso prescindir de outra.\pause
		\item Na produ\c c\~ao, efici\^encia \'e incompat\'ivel com desaproveitamento de recursos.
	\end{itemize}
\end{frame}

\begin{frame}
	\frametitle{Bem-Estar Social}
	\begin{itemize}
		\item Refere-se \`a adi\c c\~ao de todos os benef\'icios que decorrem das escolhas para todos os agentes econ\'omicos. \pause
		\item Se, a partir de uma situa\c c\~ao de efici\^encia de Pareto se puder alterar as escolhas beneficiando algum(uns) agente(s) econo\'omico(s) de forma a que o seu benef\'icio adicional compense a perda provocada noutro(s) agente(s), para garantir maior equidade por exemplo, haver\'a uma melhoria de bem-estar e trata-se de um movimento eficiente (Kaldor-Hicks) \pause
		\item O bem-estar oscial ser\'a m\'aximo quando se esgotarem todos os movimentos de Kaldor-Hicks.
	\end{itemize}
\end{frame}