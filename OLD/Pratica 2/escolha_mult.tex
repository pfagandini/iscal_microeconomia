\begin{frame}
    \frametitle{Questão 1}
    
    As preferências de um consumidor em relação a bens podem ser representadas por:

    \begin{enumerate}
           
    \item Curvas de indiferença 
    \item Funções de utilidade 
    \item Restrições orçamentais 
    \item Ambas A e B

    \end{enumerate}
    \onslide<2>{Resposta: 4}
\end{frame}
    
\begin{frame}
    \frametitle{Questão 2}
    
    Uma curva de indiferença mostra:
    
    \begin{enumerate}
    
    \item Todas as combinações de bens que fornecem ao consumidor o mesmo nível de satisfação. 
    \item Os preços de diferentes bens. 
    \item O nível mais alto de satisfação que o consumidor pode alcançar com sua renda. 
    \item Todas as combinações de bens acessíveis, dado o orçamento do consumidor.

    \end{enumerate}
    \onslide<2>{Resposta: 1}
\end{frame}
    
\begin{frame}
    \frametitle{Questão 3}
    
    A Primeira Lei de Gossen afirma que:
    \begin{enumerate}
    
    \item A utilidade marginal de um bem diminui à medida que mais dele é consumido. 
    \item Preço e quantidade demandada são inversamente proporcionais. 
    \item Os consumidores buscam maximizar a utilidade dada sua renda. 
    \item À medida que a renda aumenta, o consumo aumenta.
\end{enumerate}
\onslide<2>{Resposta: 1}
\end{frame}
    
\begin{frame}
    \frametitle{Questão 4}
    
    Com uma função de utilidade Cobb-Douglas da forma $U(x, y) = x^a y^b$, o conjunto ótimo é encontrado onde: 
    \begin{enumerate}
    
    \item A taxa marginal de substituição (TMS) é igual à razão dos preços ($p_x/p_y$). 
    \item O preço do bem $x$ é igual ao preço do bem $y$. 
    \item O consumidor gasta toda a renda disponível. 
    \item Ambas A e C
\end{enumerate}
\onslide<2>{Resposta: 4}
\end{frame}
    
\begin{frame}
    \frametitle{Questão 5}
    
    Uma restrição orçamental representa: 
    \begin{enumerate}
    
    \item O gasto máximo possível em bens que irá maximizar a utilidade. 
    \item As diferentes combinações de bens que um consumidor pode comprar, dado sua renda e preços. 
    \item As preferências ideais do consumidor. 
    \item A quantidade máxima de um bem que o consumidor poderia consumir.
\end{enumerate}
\onslide<2>{Resposta: 2}
\end{frame}

\begin{frame}
    \frametitle{Questão 6}
    
    A taxa marginal de substituição (TMS) representa: 
    \begin{enumerate}
    \item O preço de um bem em termos de outro. 
    \item A taxa na qual um consumidor está disposto a trocar um bem por outro, mantendo o mesmo nível de satisfação. 
    \item A utilidade total obtida ao consumir um conjunto de bens. 
    \item  A mudança na utilidade quando o consumo de um bem aumenta em uma unidade.
\end{enumerate}
\onslide<2>{Resposta: 2}
\end{frame}
    
    \begin{frame}
    \frametitle{Questão 7}
    
    A procura é derivada de: 
    \begin{enumerate}
    \item Maximização do excedente do consumidor 
    \item Maximização da utilidade, dados os preços e a renda. 
    \item Análise de estática comparativa. 
    \item Observação do comportamento do mercado.
\end{enumerate}
\onslide<2>{Resposta: 2}
\end{frame}
    
    \begin{frame}
    \frametitle{Questão 8}
    
    O excedente do consumidor é: 
    \begin{enumerate}
    \item A renda restante após a compra do conjunto ideal de bens. 
    \item A diferença entre o preço que um consumidor está disposto a pagar e o preço que realmente paga. 
    \item A utilidade total obtida com o consumo de bens. 
    \item Representado pela área acima da curva de demanda e abaixo do preço
\end{enumerate}
\onslide<2>{Resposta: 2}
\end{frame}
    
    \begin{frame}
    \frametitle{Questão 9}
    
    Se um consumidor possui uma função de utilidade $U(x, y) = x^{0.25} y^{0.75}$, e enfrenta preços $p_x$ e $p_y$, qual será a quantidade do bem $x$ no seu cabaz \'otimo? [Nota: $W$ representa o or\c camento dispon\'ivel.] 
    \begin{enumerate}
    \item $(0,25 W) / Px $
    \item $(0,75 W) / Px $
    \item $(0,25 W) / Py $
    \item $(0,75 W) / Py$
\end{enumerate}
\onslide<2>{Resposta: 1}
\end{frame}
    
    \begin{frame}
    \frametitle{Questão 10}
    
    Uma diminuição no preço de um bem causará: 
    \begin{enumerate}
    \item Aumento do excedente do consumidor. 
    \item Diminuição do excedente do consumidor. 
    \item O consumidor a demandar mais do bem, mesmo que não goste dele. 
    \item Nenhuma mudança no comportamento do consumidor.
\end{enumerate}
\onslide<2>{Resposta: 1}
\end{frame}